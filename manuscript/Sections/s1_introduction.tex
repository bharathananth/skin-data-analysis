% BIOLOGY OF SKIN
The skin is the largest organ of the body and its main functions are protection against bacteria, radiation or temperature from the exterior, as well as against water loss from the interior\cite{Wong2016}. It is morphologically complex and is populated by many cell types organized into three main layers: epidermis, dermis and hypodermis \cite{Plikus2015}. The skin evolved a circadian clock \cite{Allada2021} in response to the direct exposure to the rhythmic external environment to anticipate changes and to adjust its physiology accordingly. 

%With the outside world changing throughout the 24\,h day and given the direct environmental exposure of the skin, it does not come as a surprise that circadian clocks have evolved to allow the skin to anticipate the environmental shifts and adjust its physiology accordingly. In fact, diurnal rhythms are observed in multiple (if not all) cell types across all layers of skin that regulate a variety of physiological responses. %With the outside world changing throughout the 24\,h day and the skin providing the first line of defense against many environmental factors that exhibit dramatic diurnal variations such as temperature or radiation, a mechanism has evolved to allow the skin to anticipate the environmental shifts and adjust accordingly. Diurnal rhythms are observed in multiple cell types across all layers of skin. || iven the direct environmental exposure of the skin, as well as its cellular complexity, it is likely that circadian clocks may regulate a variety of distinct physiological responses depending on cell types.
\newpage
% UNDER THE CONTROL OF THE CIRCADIAN CLOCK -- EVIDENCES: Andersen, Brown, Wu, Benitah, Spoerl...

The mammalian circadian clock is a hierarchical network with the central clock in the suprachiasmatic nucleus (SCN) and peripheral clocks in many tissues including the skin. The cell-autonomous molecular ``core'' clock \cite{Takahashi2017, Dibner2010} consists of a number of interlocked transcriptional-translational negative feedback loops. Core clock genes \textit{CLOCK} and \textit{BMAL1} induce the expression of their own inhibitors, \textit{PER} and \textit{CRY} genes. Once translated, PER and CRY proteins form large complexes that travel back to the nucleus to repress CLOCK and BMAL1, thus repressing their own transcription and thereby creating self-sustained 24\,h rhythms in gene and protein expression. In mammals, core clock components that are also transcription factors act at \textit{cis}-regulatory sequences to drive rhythmic expression of a large number of output genes (almost 10\% of all genes) in a cell-autonomous and tissue-specific manner \cite{Zhang2014, Ruben2018}.

The circadian gene expression profile of the skin remains nevertheless incompletely characterized. The presence of a skin circadian clock in humans was first inferred from circadian rhythms in biophysical skin parameters, such as sebum secretion \cite{Burton1970}, water loss \cite{Spruit1971} or response to allergens \cite{Reinberg1965}. At the turn of the century, rhythmic expression of selected core clock genes in the skin were described in humans \cite{Bjarnason2001} and mice \cite{Oishi2002}. Over time, circadian expression of core clock genes was recorded in several skin cell types, including epidermal and hair follicle keratinocytes, dermal fibroblasts and melanocytes \cite{Zanello2000, Kawara2002, Brown2005, Brown2008,Spoerl2011}. Sp\"orl et al. (co-authors of the present study) \cite{Spoerl2012} performed the first high-throughput analysis of circadian gene expression in the skin. This study identified $\sim$300 circadian genes in one layer (epidermis). More recently, Wu et al. \cite{Wu2018,Wu2020} identified $\sim$100-150 circadian genes each in the epidermis and dermis. However, both these microarray studies provide limited insight into circadian gene expression in human skin due to insufficient number of samples over one circadian cycle. To more thoroughly describe the impact of the human skin clock, we identified circadian genes in the two prominent skin layers, epidermis and dermis. To assess whether the complex and heterogeneous skin also results in a cell type-/layer- specific clock, we compared the circadian gene expression across layers. 

%At least 1400 genes involved in different functions show circadian expression changes in mouse skin \cite{Plikus2015}, suggesting that the circadian clock may, indeed, influence various aspects of skin physiology, including susceptibility to UV-induced DNA damage \cite{Geyfman2012, Wang2017}, barrier recovery \cite{Yosipovitch2004}, trans-epidermal water loss \cite{Yosipovitch1998}, sebum secretion, skin temperature or skin pH \cite{LeFur2001}.  While it is known that the central clock influences circadian rhythms within skin \cite{Tanioka2009}, evidence in the last years has shown that clock regulation in skin is not just an output of the central suprachiasmatic nucleus (SCN), but rather, skin itself, like most organs, harbors robust intrinsic clocks. Already more than 10 years ago, circadian oscillations were found to be present in several skin cell types, including epidermal and hair follicle keratinocytes, dermal fibroblasts and melanocytes \cite{Zanello2000, Bjarnason2001, Kawara2002, Oishi2002, Brown2005, Brown2008, Spoerl2011}. Nevertheless, on a molecular level, it is still unclear what are the differences between clocks in different skin layers and how such clocks might contribute to rhythmic skin function. \\ %\textcolor{red}{What about the Janich-Benitah stories? Also diseases: psoriasis, skin cancer -- Gaddameehdi2011, ageing... (reviewed in \href{https://pubmed.ncbi.nlm.nih.gov/34535902/}{\textbf{Duan-Andersen2021}) -- maybe talk about diseases in discussion?} Circadian timing mechanisms are also sensitive to day length and temperature, and therefore circadian clock mechanisms are candidates for the regulation of seasonal phenomena within the skin. Interestingly, several skin diseases do exhibit seasonal change in severity \href{https://pubmed.ncbi.nlm.nih.gov/18755376/}{Weiss et al 2008}}\\

% WHAT IS MISSING 1? WHAT ARE REMAINING ISSUES 1? good set of biomarkers
As one of the few accessible tissue clocks, skin samples could be used for circadian phenotyping of humans. Evidence is gradually accumulating that therapeutic efficacy and the degree of side effects are time-of-day-dependent\cite{Montaigne2018,Dallmann2016,Long2016}. Such observations are likely to grow, since 50\% of all drugs target the product of a circadian gene \cite{Ruben2018, Zhang2014}. One key challenge to implementing time-of-day-aware `circadian medicine' strategies is that internal clocks of humans are heterogeneous. Since rhythms in human physiology are determined by \textit{internal} clock time and not on time according to the external environment, circadian studies in humans ought to record and present results relative to the internal phase of entrainment (termed chronotype) of subjects. This internal clock time in turn depends on genetic factors, \cite{Hsu2015, Brown2008}, age and sex \cite{Roenneberg2007}, level of light exposure \cite{Stothard2017, Wright2013}, the season \cite{Stothard2017, Allebrandt2014} and on the local time-zone \cite{Roenneberg2007}. Thus, circadian treatments need to be personalized to an individual's clock.  To evaluate the utility of skin for circadian phenotyping, we used our comprehensive gene expression profiles to identify biomarkers for circadian phase in each layer. %Evidence is gradually accumulating of time-of-day dependence of therapeutic efficacy and degree of side-effects\cite{Montaigne2018,Dallmann2016,Long2016}. 



%But skin, besides providing additional knowledge into its circadian biology, also represents a potential source for circadian biomarker discovery. Around 50\% of all drugs target the product of a circadian gene \cite{Anafi2017}. Moreover, therapeutic outcomes such as survival after surgical procedures \cite{Montaigne2018}, efficacy and tolerance of chemotherapy \cite{Dallmann2016} or antibody response to vaccination \cite{Long2016} all vary diurnally. For this reason, a practical measure for circadian phase is needed if we want circadian medicine to influence health in any way. A key limitation in the implementation of chronotherapeutic approaches is the fact that humans are heterogeneous with respect to the timing of their internal clocks \cite{Roenneberg2007}. Humans exhibit different phases of entrainment or chronotypes. This is, the alignment phase angle of one's physiological and behavioral rhythms with respect to the environmental changes from individual to individual and it shows a normal distribution ranging from very early chronotypes (larks) to very late chronotypes (owls) \cite{Roenneberg2007}. The human chronotype is usually assessed by questionnaires that examine sleeping habits \cite{Horne1976, Roenneberg2003}, which are normally not objective, or with strategies that require multiple measurements under controlled conditions. The current cold standard tool for assessing human circadian phase is the dim-light melatonin onset (DLMO) assay, which requires a subject to sit in a dim room for repeated saliva sample collection, a difficult practice to standardize and perform at large scales and burdensome for clinical practice. DLMO is considered the marker of SCN phase and locomotor activity \cite{Pandi2007, Laing2017}. This raises several important and open questions: Is the DLMO phase aligned with peripheral clocks? Are there better sources of circadian biomarkers? What should a good biomarker measure? \\

% WHAT IS MISSING 2? WHAT ARE REMAINING ISSUES 2? internal time issue and sources of variation
%An individual's (or a tissue's) circadian time refers to the phase of its \textit{internal }biological clocks. This inner phase of entrainment (or chronotype) depends on many factors. It has a genetic basis \cite{Hsu2015, Brown2008}, it is age- and sex- dependent \cite{Roenneberg2007}, depends on the light exposure level \cite{Stothard2017, Wright2013}, on the season \cite{Stothard2017, Allebrandt2014} and on the time-zone where that subject is located \cite{Roenneberg2007}. Thus, any algorithm that aims to assess the phase of circadian biomarkers should take these factors into account in order to correct external time (i.e. time of sampling) to \textit{internal} circadian time \cite{Laing2019}. For this reason, when aiming at determining circadian time in a specific human tissue, it is important to obtain a dataset in which not only the time of sampling is recorded, but also as much information from the subjects as possible. This way we might be able to more accurately control and correct wall (external) time to \textit{internal }time. \\

% WHAT DO WE PROPOSE AND HOW DO WE ADDRESS IT? WHAT HAVE WE FOUND?
We measured gene expression in a small cohort of young, healthy adults of both sexes, who maintained their natural yet regular sleep schedules using microarrays. We first quantified the circadian expression expected in the general population in the epidermis and dermis. In contrast to previous studies on skin, we present our results with respect to internal time of subjects; our study design included chronotyping of subjects. We then analyzed the layer-specificity of \textit{population} circadian rhythms. The population circadian rhythms is our best guess of rhythms in an individual in the population. However, each individual in the population will have rhythms that differ from the population rhythm. Therefore, we next evaluated how much individual circadian rhythms deviate from the population rhythms in both layers. Finally, we identify a small set of biomarkers for circadian phase in each layer.


%obtained through whole-genome microarray analysis of \textcolor{red}{suction-blister} skin, where 5 females and 6 males were taken a skin biopsy every 4\,h for 24\,h. In order to control for as many external factors as possible, we gathered meta-data about the subjects including time at which they go to bed during weekdays and weekends. We calculated their mid sleep time (MST) and used this value to correct wall time to internal time. Thus, our results report about clock-controlled gene expression in human dermis and epidermis with respect to internal time, with differences in chronotypes being taken into consideration. We found $\sim$1400 rhythmic transcripts in at least one of the layers and that $\sim$280 transcripts are shared between dermis and epidermis, although with some layer-specific differences despite their physical proximity: epidermis displays higher amplitude rhythms and earlier phases compared to dermis. Because of the meta-data availability, we quantified the sources of variation and assessed how layers, subjects, time, mid sleep time (our proxy for chronotype) and sex contribute to variability in mean expression or to circadian-specific differences. We found that magnitude of clock-controlled genes varies largely between subjects (and layers), while almost no magnitude- or circadian-specific variance could be attributed to differences in chronotype. Lastly, we identified a set of time-telling genes that can assess circadian phase with high accuracy in human dermis and epidermis. The novelty of our approach, in our opinion, is the fact that we corrected sampling time to internal time. It is this what should be taken into account if any evaluation of circadian phase of biomarkers is desired, as circadian time refers to the phase of \textit{internal} biological clocks.
%%%%%%%%%%%%%%%%%%%%%%%%%%%%%%%%%%%%%%%%%%%%%%%%%%%%%%%%%%%%%%%%%%%%%%%%%%%%%%%%%%%%%%%%%%%%%%%%
%%%%%%%%%%%%%%%%%%%%%%%%%%%%%%%%%%%%%%%%%%%%%%%%%%%%%%%%%%%%%%%%%%%%%%%%%%%%%%%%%%%%%%%%%%%%%%%%

%\begin{itemize}
	% CHECK UP
	% 1) What is the problem:	
	% 2) What has been done before and what and how they found it.	
	% 3) What are the remaining issues and what stuff are missing.
	% 4) What do we propose? 
	% 5) What have we done and found?
%\end{itemize}
