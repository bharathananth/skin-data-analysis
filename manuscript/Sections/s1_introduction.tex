% BIOLOGY OF SKIN
The skin is the largest organ of the body and one of its main functions is protection against bacteria, radiation or temperature from the exterior, as well as against water loss from the interior. Morphologically, the skin is complex and is populated by many cell types including epidermal keratinocytes, melanocytes, dermal fibroblasts, as well as many appendage-specific cell types, such as hair follicle keratinocytes, sebocytes, and eccrine gland cells \cite{Plikus2015}. This heterogeneous population is organized into several structures and compartments to fulfill a range of tasks, such as water loss prevention, sensation and hormone synthesis, among others \cite{Wong2016, Zouboulis2009}. \textcolor{red}{In humans and mice, the skin consists of three main layers: epidermis, dermis and hypodermis.} In response to the direct exposure to the rhythmic external environment, the skin evolved a circadian clock \cite{Allada2021} to anticipate environmental shifts and to adjust its physiology accordingly. 

%With the outside world changing throughout the 24\,h day and given the direct environmental exposure of the skin, it does not come as a surprise that circadian clocks have evolved to allow the skin to anticipate the environmental shifts and adjust its physiology accordingly. In fact, diurnal rhythms are observed in multiple (if not all) cell types across all layers of skin that regulate a variety of physiological responses. %With the outside world changing throughout the 24\,h day and the skin providing the first line of defense against many environmental factors that exhibit dramatic diurnal variations such as temperature or radiation, a mechanism has evolved to allow the skin to anticipate the environmental shifts and adjust accordingly. Diurnal rhythms are observed in multiple cell types across all layers of skin. || iven the direct environmental exposure of the skin, as well as its cellular complexity, it is likely that circadian clocks may regulate a variety of distinct physiological responses depending on cell types.
\newpage
% UNDER THE CONTROL OF THE CIRCADIAN CLOCK -- EVIDENCES: Andersen, Brown, Wu, Benitah, Spoerl...

The effect of the skin circadian clock was first observed as circadian rhythms in biophysical skin parameters, such sebum secretion \cite{Burton1970}, water loss \cite{Spruit1971} or response to allergens in humans \cite{Reinberg1965}. At the turn of the century, daily changes in the expression of circadian genes in the skin were described in humans \cite{Bjarnason2001} and mice \cite{Oishi2002}. Over time, circadian gene expression was recorded in several skin cell types, including epidermal and hair follicle keratinocytes, dermal fibroblasts and melanocytes \cite{Zanello2000, Kawara2002, Brown2005, Brown2008, Spoerl2011}.  While it was initially assumed that circadian clock in the skin was dependent and dominated by the central suprachiasmatic nucleus (SCN) clock in the brain \cite{Tanioka2009}, it is now clear that the skin clock functions independently of the circadian clock network \cite{Welz2019} \todo{If this is true, why then is the skin a good source of biomarkers for circadian/biological clock timing??}.

The mammalian circadian clock is a hierarchical network with the central clock in the SCN and peripheral clocks in many tissues \cite{}. The cell-autonomous molecular clock in all tissues \cite{Takahashi2017, Dibner2010} consists of a number of interlocked transcriptional-translational negative feedback loops. The transcription factors CLOCK and BMAL1 induce the expression of their own inhibitors, \textit{PER} and \textit{CRY} genes. When translated, PER and CRY proteins form large complexes that travel back to the nucleus to repress CLOCK and BMAL1, thus repressing their own transcription and thereby creating self-sustained 24\,h rhythms in gene and protein expression. Components of the core clock that are also transcription factors act at cis-regulatory sequences to drive  rhythmic expression of a large number of output genes (almost 10\% of all genes) in a cell-autonomous and tissue-specific manner in both mice \cite{Zhang2014} and humans \cite{Ruben2018}. In the context of the complex and heterogenous skin, it remains unclear how and what extent the circadian clock varies in a cell-type and layer-specific manner in the skin.

%At least 1400 genes involved in different functions show circadian expression changes in mouse skin \cite{Plikus2015}, suggesting that the circadian clock may, indeed, influence various aspects of skin physiology, including susceptibility to UV-induced DNA damage \cite{Geyfman2012, Wang2017}, barrier recovery \cite{Yosipovitch2004}, trans-epidermal water loss \cite{Yosipovitch1998}, sebum secretion, skin temperature or skin pH \cite{LeFur2001}.  While it is known that the central clock influences circadian rhythms within skin \cite{Tanioka2009}, evidence in the last years has shown that clock regulation in skin is not just an output of the central suprachiasmatic nucleus (SCN), but rather, skin itself, like most organs, harbors robust intrinsic clocks. Already more than 10 years ago, circadian oscillations were found to be present in several skin cell types, including epidermal and hair follicle keratinocytes, dermal fibroblasts and melanocytes \cite{Zanello2000, Bjarnason2001, Kawara2002, Oishi2002, Brown2005, Brown2008, Spoerl2011}. Nevertheless, on a molecular level, it is still unclear what are the differences between clocks in different skin layers and how such clocks might contribute to rhythmic skin function. \\ %\textcolor{red}{What about the Janich-Benitah stories? Also diseases: psoriasis, skin cancer -- Gaddameehdi2011, ageing... (reviewed in \href{https://pubmed.ncbi.nlm.nih.gov/34535902/}{\textbf{Duan-Andersen2021}) -- maybe talk about diseases in discussion?} Circadian timing mechanisms are also sensitive to day length and temperature, and therefore circadian clock mechanisms are candidates for the regulation of seasonal phenomena within the skin. Interestingly, several skin diseases do exhibit seasonal change in severity \href{https://pubmed.ncbi.nlm.nih.gov/18755376/}{Weiss et al 2008}}\\

% WHAT IS MISSING 1? WHAT ARE REMAINING ISSUES 1? good set of biomarkers
As the exterior-most organ in possession of a clock, the skin is also an accessible source of material for phenotyping the circadian clock in humans. There is gradually accumulating evidence that there is a time-of-day dependence of therapeutic efficacy and degree of side-effects\cite{Montaigne2018,Dallmann2016,Long2016}. Such observations are likely to grow, since 50\% of all drugs target the product of a circadian gene \cite{Ruben2018, Zhang2014}. One of the key challenges to implementing such time-of-day aware `circadian medicine' strategies is that humans and their clocks are heterogeneous. An important and often overlooked detail is that rhythms in human physiology are determined by the \textit{internal} clock of humans and not on the time according to the external environment. This internal clock time depends on multiple factors, such as genetic factors, \cite{Hsu2015, Brown2008}, age and sex \cite{Roenneberg2007}, level of light exposure \cite{Stothard2017, Wright2013}, the season \cite{Stothard2017, Allebrandt2014} and on the local time-zone \cite{Roenneberg2007}. Thus, circadian treatments need to be personalized to an individual's clock. Moreover, this suggests that to draw meaningful conclusions circadian studies in humans must record and present results relative to the internal phase of entrainment (termed chronotype) of subjects.

%But skin, besides providing additional knowledge into its circadian biology, also represents a potential source for circadian biomarker discovery. Around 50\% of all drugs target the product of a circadian gene \cite{Anafi2017}. Moreover, therapeutic outcomes such as survival after surgical procedures \cite{Montaigne2018}, efficacy and tolerance of chemotherapy \cite{Dallmann2016} or antibody response to vaccination \cite{Long2016} all vary diurnally. For this reason, a practical measure for circadian phase is needed if we want circadian medicine to influence health in any way. A key limitation in the implementation of chronotherapeutic approaches is the fact that humans are heterogeneous with respect to the timing of their internal clocks \cite{Roenneberg2007}. Humans exhibit different phases of entrainment or chronotypes. This is, the alignment phase angle of one's physiological and behavioral rhythms with respect to the environmental changes from individual to individual and it shows a normal distribution ranging from very early chronotypes (larks) to very late chronotypes (owls) \cite{Roenneberg2007}. The human chronotype is usually assessed by questionnaires that examine sleeping habits \cite{Horne1976, Roenneberg2003}, which are normally not objective, or with strategies that require multiple measurements under controlled conditions. The current cold standard tool for assessing human circadian phase is the dim-light melatonin onset (DLMO) assay, which requires a subject to sit in a dim room for repeated saliva sample collection, a difficult practice to standardize and perform at large scales and burdensome for clinical practice. DLMO is considered the marker of SCN phase and locomotor activity \cite{Pandi2007, Laing2017}. This raises several important and open questions: Is the DLMO phase aligned with peripheral clocks? Are there better sources of circadian biomarkers? What should a good biomarker measure? \\

% WHAT IS MISSING 2? WHAT ARE REMAINING ISSUES 2? internal time issue and sources of variation
%An individual's (or a tissue's) circadian time refers to the phase of its \textit{internal }biological clocks. This inner phase of entrainment (or chronotype) depends on many factors. It has a genetic basis \cite{Hsu2015, Brown2008}, it is age- and sex- dependent \cite{Roenneberg2007}, depends on the light exposure level \cite{Stothard2017, Wright2013}, on the season \cite{Stothard2017, Allebrandt2014} and on the time-zone where that subject is located \cite{Roenneberg2007}. Thus, any algorithm that aims to assess the phase of circadian biomarkers should take these factors into account in order to correct external time (i.e. time of sampling) to \textit{internal} circadian time \cite{Laing2019}. For this reason, when aiming at determining circadian time in a specific human tissue, it is important to obtain a dataset in which not only the time of sampling is recorded, but also as much information from the subjects as possible. This way we might be able to more accurately control and correct wall (external) time to \textit{internal }time. \\

% WHAT DO WE PROPOSE AND HOW DO WE ADDRESS IT? WHAT HAVE WE FOUND?
In this work, we quantify the \emph{population} circadian gene \todo{The mean expression of circadian genes (henceforth refers to genes with circadian expression)} expression in the dermis and epidermis across a healthy human cohort with respect to internal time and contrast this average circadian gene expression between the two layers. Two-thirds of the circadian gene expression was shared between these physically proximal layers, but a third of the rhythmic genes showed layer-specific circadian expression. This layer-specificity was manifested as the rhythms in the epidermis having higher amplitude rhythms and earlier phases compared to the dermis\todo{is this only based on DR genes?}. \todo{test whether this population circadian rhythm is affected by the lack of consideration of internal time (as is typically the case). However, these results did not differ when the analysis was repeated with respect to external time. Put this in intro? Does is weaken our paper?} We next characterize how much individual circadian rhythms deviate from the population rhythms in both layers. We found .... Finally, we identify circadian biomarkers for internal clock phase in both layers and ...


%obtained through whole-genome microarray analysis of \textcolor{red}{suction-blister} skin, where 5 females and 6 males were taken a skin biopsy every 4\,h for 24\,h. In order to control for as many external factors as possible, we gathered meta-data about the subjects including time at which they go to bed during weekdays and weekends. We calculated their mid sleep time (MST) and used this value to correct wall time to internal time. Thus, our results report about clock-controlled gene expression in human dermis and epidermis with respect to internal time, with differences in chronotypes being taken into consideration. We found $\sim$1400 rhythmic transcripts in at least one of the layers and that $\sim$280 transcripts are shared between dermis and epidermis, although with some layer-specific differences despite their physical proximity: epidermis displays higher amplitude rhythms and earlier phases compared to dermis. Because of the meta-data availability, we quantified the sources of variation and assessed how layers, subjects, time, mid sleep time (our proxy for chronotype) and sex contribute to variability in mean expression or to circadian-specific differences. We found that magnitude of clock-controlled genes varies largely between subjects (and layers), while almost no magnitude- or circadian-specific variance could be attributed to differences in chronotype. Lastly, we identified a set of time-telling genes that can assess circadian phase with high accuracy in human dermis and epidermis. The novelty of our approach, in our opinion, is the fact that we corrected sampling time to internal time. It is this what should be taken into account if any evaluation of circadian phase of biomarkers is desired, as circadian time refers to the phase of \textit{internal} biological clocks.
%%%%%%%%%%%%%%%%%%%%%%%%%%%%%%%%%%%%%%%%%%%%%%%%%%%%%%%%%%%%%%%%%%%%%%%%%%%%%%%%%%%%%%%%%%%%%%%%
%%%%%%%%%%%%%%%%%%%%%%%%%%%%%%%%%%%%%%%%%%%%%%%%%%%%%%%%%%%%%%%%%%%%%%%%%%%%%%%%%%%%%%%%%%%%%%%%

%\begin{itemize}
	% CHECK UP
	% 1) What is the problem:	
	% 2) What has been done before and what and how they found it.	
	% 3) What are the remaining issues and what stuff are missing.
	% 4) What do we propose? 
	% 5) What have we done and found?
%\end{itemize}
