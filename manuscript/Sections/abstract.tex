The skin is the largest human organ with a circadian clock that regulates its function. Although circadian rhythms in specific functions are known, rhythms in the proximal clock output, gene expression, in human skin have not been thoroughly explored. This work reports circadian gene expression in two skin layers, epidermis and dermis, in a young, healthy cohort that maintained natural, regular sleep schedules. 10\% of the expressed genes showed rhythms at the population level, of which only a third differed between the two layers. Broadly, magnitudes of circadian genes were consistent across subjects in each layer. Amplitude and phases of circadian genes, however, varied more across subjects than layers, with amplitude being more variable than phases. Amplitudes in the epidermis were larger and more subject-variable, while they were smaller and more consistent in the dermis. Core clock genes were similar across layers at the population-level, but were heterogeneous in the their variability across subjects. We used this data to identify small sets of biomarkers for internal clock phase in each layer, which consisted of layer-specific non-core clock genes. This work provides a valuable resource to advance our understanding of human skin and leverage the potential of circadian medicine as well as a novel methodology to quantify sources of variability in human circadian rhythms.



%Skin is the largest organ in the human body and serves as an important protective barrier. With the direct exposure to strong day-dependent changes in the environment, recent work has unraveled an important role of the circadian clock in regulating skin functions. We present in this paper a novel human dataset, obtained through whole-genome microarray analysis of skin punch biopsies, in which we performed a comprehensive analysis of the circadian transcriptome in human dermis and epidermis with respect to internal time. We found that a fifth of of the circadian skin transcriptome is shared between both layers, while a large part is layer-specific. Despite such close physical proximity, amplitude and phase seem to be different in both layers, with epidermis exhibiting higher amplitude rhythms and earlier circadian phases. We quantified the sources of variation and assessed how layers, subjects, time, mid sleep time (our proxy for chronotype and thus internal time) and sex contribute to variability in mean expression (magnitude) or to circadian-specific differences. We found that magnitude of clock-controlled genes varies largely between subjects (and layers), while almost no magnitude- or circadian-specific variance could be attributed to differences in chronotype. Lastly, we identified a set of time-telling genes that can assess circadian phase with high accuracy in human dermis and epidermis. What, in our opinion, distinguishes our approach from previous skin transcriptomic studies in human skin is the fact that we have used mid sleep time to correct the sampling time to internal time, which should be the aim of any study that aims to report about circadian (internal) time. 
