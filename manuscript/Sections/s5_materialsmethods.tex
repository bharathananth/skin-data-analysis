\subsection*{Experimental design and collection of human skin samples}
The study to obtain skin punch biopsies was approved by the Beiersdorf AG Legal Review Board. Tissue samples were collected according to the recommendations of the Declaration of Helsinki and according to applicable laws for a non-drug study. All donors provided written and informed consent. The study was performed at the study center of Beiersdorf AG. Eleven healthy volunteers (six males, five females, aged 20-30\,years) participated in the study. Individual chronotypes were assessed using the Munich Chronotype Questionnaire (MCTQ) by calculating the \textit{corrected} mid sleep time, this is, the mid sleep time on free days adjusted for the sleep-debt accumulated during the workweek \cite{Vetter2021}. \textcolor{red}{meals? physical activity?} All information about the subjects is provided in Supplementary Table \ref{tab:supptab1}. \\ %All subjects were provided with regularly scheduled meals without specific dietary restrictions and were not allowed to perform any exhaustive physical activities. No food was served after 10 PM. Wu-Hogenesch 2018 https://www.pnas.org/content/pnas/suppl/2018/10/29/1809442115.DCSupplemental/pnas.1809442115.sapp.pdf

3\,mm punch biopsies were obtained from the upper back for seven time points over a period of 24\,h (8\,AM, 12\,PM, 4\,PM, 8\,PM, 12\,AM, 4\,AM and 8\,AM the following day). Skin biopsies were subsequently incubated in PBS at 55\,\textsuperscript{o}C for 3\,min to separate epidermis and dermis. Tissue samples were then frozen in liquid nitrogen and stored at -80\,\textsuperscript{o}C. \textcolor{red}{RNA extraction and quality control from punch-biopsies was performed by Miltenyi Biotec using the TRIzol method. Linear amplification and labeling of RNA and hybridization of Agilent Whole Genome Oligo Microarrays 4x44k (Agilent Technologies) using 1.2--1.65\,\textmu g of Cy3-labeled cRNA was performed by Miltenyi Biotec, essentially as reported in \cite{Duggan1999}.} 

\subsection*{Gene expression analysis}
The microarray gene expression analysis was conducted in R. The \textcolor{red}{RMA} (Robust Multichip Average) algorithm was used to pre-process and extract expression profiles from the raw CEL files. Transcripts were annotated with ENSEMBL and ENTREZ IDs using \textcolor{red}{Agilent ``Human Genome, Whole'' annotation data (hgug4112a.db, v3.2)}. The raw gene expression data has been deposited in the Gene Expression Omnibus (GEO) under the accession number \textcolor{red}{GSEblabla}. \\

Raw data of the hybridized microarrays were normalized using the Bioconductor R-Project package Linear Models for Microarray Data (\texttt{limma}). Background correction was performed using the \texttt{normexp} function; normalization between the different arrays was done with the \texttt{quantile} function. Non-annotated probes as well as control probes were filtered out. Genes were considered to be expressed if the \textcolor{red}{signal across the 7 time points was well above the background signal in at least 50\% of the microarrays}. Lowly-expressed genes were removed from further analysis. The expression of genes annotated by multiple ENSEMBL IDs was averaged across replicates using the \texttt{avereps} function from \texttt{limma}, resulting in a filtered dataset containing data for 11578 transcripts. \\

Principal Component Analysis (PCA) was performed in order to remove outliers. The expressed genes nicely organized in two clusters in PCA space, separated by skin layer (data not shown). Nevertheless, the epidermis sample from subject 109 taken at 8\,AM the following day did not cluster with the rest of epidermal samples and for this reason was removed from further analyses. 

\subsection*{Rhythmicity analysis and functional annotation of circadian gene lists}
To detect transcripts exhibiting rhythmic behavior with a 24\,h period in their expression, successive filtering steps were applied. For detection of circadian rhythmicity, we applied \textcolor{red}{cosinor analysis} and we tested only \textit{one} null hypothesis (in the lines of \cite{Pelikan2021}), namely that rhythms in dermis and epidermis are the same. Each microarray was assigned to its corresponding skin layer and time-group (i.e., 8\,AM, 12\,PM, 4\,PM, 8\,PM, 12\,AM, 4\,AM or 8\,AM the following day). In the analyses where \textit{internal} time was used, sampling (wall) time was corrected to internal time in each subject by calculating the difference in mid sleep time (MST) to a reference subject (that with median MST), and subtracting this value to wall time. A linear model was then fitted to the expression data for each probe using the \texttt{lmFit} function from the \texttt{limma} package. Subsequently, a moderated F-test of the empirical Bayes statistics for differential expression (\texttt{eBayes} function from the \texttt{limma} package) was applied to test differential gene expression between the respective groups for statistical significance. The \texttt{topTable} function from \texttt{limma} was used to summarize the linear model fit object produced by \texttt{lmFit} and processed by \texttt{eBayes}. \\ %An offset of 20 was added to stabilize the background and, subsequently, signals were log2-transformed. Genes were considered to be expressed if the background-subtracted signal was above 2.6 times the SD of the background signal in at least 75\,\% of the microarrays. 

Acrophases and relative amplitudes were estimated from the cosinor analysis. In order to identify significant circadian transcripts, $p$ values were adjusted for multiple testing by the Benjamini-Hochberg (BH) method, thus controlling the expected false discovery rate (FDR). An FDR cutoff below 0.05 and a relative amplitude threshold of 0.26 were used to filter for expressed trancripts with significant rhythms in dermis or epidermis. \\

Genes showing significant diurnal expression patterns were tested for over-representation in Gene Ontology (GO) terms and Kyoto Encyclopedia of Genes and Genomes (KEGG) pathways against the background of expressed transcripts in each skin layer \textcolor{red}{(unless otherwise stated)} by applying the \texttt{enrichGO} and \texttt{enrichKEGG} functions included in the \texttt{clusterProfiler} package. \\

Circadian GO terms and KEGG pathways were determined by Phase Set Enrichment Analysis \cite{Zhang2016} based on the sets of circadian transcripts in each layer. Gene sets were downloaded from the Molecular Signatures database (MSigDB) C2 (KEGG gene sets) and C5 (GO:BP terms) \cite{Subramanian2005}. Sets containing fewer than five circadian transcripts were excluded from the analysis. The Kuiper test was used to identify circadian gene sets by comparing the acrophases of all circadian transcripts (rounded to the full hour) belonging to each gene set to a uniform background distribution and by testing for nonuniformity ($q<0.05$ for GO terms, $q<0.25$ for KEGG pathways).\\

A full description of all versions and references of R packages used is provided in Supplementary Table \ref{tab:supptab2}. 

\subsection*{Assessment of variation in human epidermal and dermal rhythmic genes using the \texttt{variancePartition} pipeline}
To study the different sources of variation in mean and circadian expression of the rhythmic genes in dermis and epidermis and to quantify their contribution, we used the open-source \texttt{variancePartition} package (available in Bioconductor) \cite{Hoffman2016}. \texttt{variancePartition} provides a general framework for understanding drivers of variation in gene expression experiments with complex designs. It uses a linear mixed model to partition the variation attributable to multiple variables in the data. When providing an R formula indicating which meta-data variables to consider, \texttt{variancePartition} assesses the contribution of each meta-data variable to variation in gene expression by using a multiple regression model. This way, the effect of each meta-data variable is analyzed but while jointly accounting for all the others. Importantly, \textit{external} and not internal time was used in the \texttt{variancePartition} analyses in order to be able to model time as a random effect (continuous variables like \textit{internal} time cannot be modeled as random effects in \texttt{variancePartition}). \\

We assessed the drivers of variation in different sets of transcripts: in the total number of rhythmic genes in at least one layer ($\sim1400$ transcripts) in Figure \ref{fig:fig2}A, B; in the total number of rhythmic genes only in dermis ($\sim500$ transcripts, Figure \ref{fig:fig2}C, E) and only in epidermis ($\sim1200$ transcripts, Figure \ref{fig:fig2}D, F); in 1000 non-rhythmic genes (Figure \ref{fig:suppfig3}A); in the rhythmic core-clock genes (Figure \ref{fig:suppfig3}B) and in the 25 and 15 time-telling genes that we identified using \texttt{ZeitZeiger }in dermis and epidermis, respectively (Figure \ref{fig:suppfig5}B).

\subsection*{Identification of potential biomarkers of molecular skin phase using \texttt{ZeitZeiger}}
From the multiple methods to estimate internal circadian time, we used ZeitZeiger \cite{Hughey2016} (available in Bioconductor) to identify skin biomarkers of circadian phase. We tested three sets of predictors that differed on the dataset used (the whole set of expressed genes in at least one layer versus expressed genes in dermis and epidermis separately). The predicted variable was, in both cases, internal time. To evaluate the performance of the different predictors, we followed a leave-one-subject-out cross-validation approach in the lines of \cite{Hughey2016, Wittenbrink2018}. To do this, predictors are trained with data from all subjects except one and internal time from the subject who is left-out is predicted. The process is iterated along all subjects and for different values of the two main parameters of ZeitZeiger, \texttt{sumabsv} and \texttt{nSPC}. The first parameter \texttt{sumabsv} controls how many genes form each sparse principal component (SPC) and the second parameter, \texttt{nSPC}, controls how many SPCs are used for prediction. Large values of either parameter result in more genes being needed for prediction. For each set of values of \texttt{sumabsv} and \texttt{nSPC} from the leave-one-subject-out cross-validation, we calculated the median absolute difference between the predicted and the observed internal time stamp across all subjects. We refer to this parameter as median absolute error (MAE), and it serves as a measure of accuracy of the prediction: the lower the error, the better the prediction. 

%In terms of optimal number of SPCs, \textcolor{red}{all} predictors performed comparably well, since the MAE did not decrease much after 2 SPCs. However, the optimal value of \texttt{sumabsv} and, consequently, the number of genes used for prediction, differed in each approach. Running ZeitZeiger in the whole set of skin-expressed genes (in either dermis \textit{or} epidermis) resulted in a minimum MAE of $\sim0.08$ for \texttt{nSPC}=2 and \texttt{3} (Figure \textcolor{red}{suppfig4A}) with 38 genes needed for prediction (Figure \textcolor{red}{suppfig4B}). On the other hand, running ZeitZeiger in each skin layer separately performed better, as seen by the lower MAE and the lower number of genes needed to predict internal time: 2 SPCs and \texttt{sumabsv}=2 resulted in a MAE of 0.04 (Figure \ref{fig:fig3}A), and 15-25 genes needed for prediction (Figure \ref{fig:fig3}B, time series in \textcolor{red}{suppfig5A}).\\




