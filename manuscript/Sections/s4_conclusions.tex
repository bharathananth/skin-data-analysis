\normalsize
We presented the most complete description to date of the transcriptional output of the circadian clock in two human skin layers. We reported both how average gene expression rhythmically varies in the population and the inherent variability in these average rhythms. Our consideration of internal time in the analysis makes our results applicable to humans regardless of chronotype. Not only does our work provide a comprehensive resource of circadian gene expression and circadian biomarkers for phase in human skin, but also provides a methodology to describe human circadian rhythms in a population.

%In summary, we have identified the circadian transcriptome in human dermis and epidermis and reported in differences between these two closely located layers. We have quantified the variation in mean expression of rhythmic genes, and the circadian-specific variation across subjects and skin layers, and found that clock-controlled gene magnitude varies largely between subjects. Lastly, we have identified a set of 15 and 25 genes that can assess circadian phase with high accuracy in human epidermis and dermis, respectively. What, in our opinion, distinguishes our approach from previous clock-controlled gene expression studies in human skin is the fact that we have used MST to correct the sampling time, which should be the goal of human studies that aim to predict circadian (internal) time. 
%\subsection*{Our findings in a nutshell}
%\begin{enumerate}
	%\item We have observed differences between both layers in terms of which genes are rhythmic, amplitude and phase distributions of rhythmic genes
	%\item We are not the first ones describing the skin transcriptome, but we are the first ones describing it with respect to \textit{internal} time. Moreover, our approach was to look at the source of variation in the population. 
	%\item We have quantified the variation in expression of rhythmic genes, and circadian variation of rhythmic genes.
	%\begin{enumerate}
	%	\item Variability across layers produces the largest source of variation in mean expression (of rhythmic genes)
	%	\item Inter-subject variation in mean expression is larger than the inter-time variation (what we call ``common'' circadian variation)
	%	\item We have observed that the layer or subject effects on variation across time (inter-layer or inter-subject circadian variation, respectively) are smaller than the common circadian variation
	%	\item Thus, we expect to find ``good'' time-telling genes even across layers! $\rightarrow$ \textcolor{red}{reason to do ZZ in all samples together}
	%\end{enumerate}
	%\item We identified a small set of genes that predict phase from human skin samples. \textcolor{red}{-- combine with one sample like Wu-Hogenesch? Somehow validate this? I have to read more to see if this is in any way doable.}
	%\item we can better predict internal time if we do dermis vs epidermis separately, although that poses a complication for practical reasons.
%\end{enumerate}

